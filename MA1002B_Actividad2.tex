% Options for packages loaded elsewhere
\PassOptionsToPackage{unicode}{hyperref}
\PassOptionsToPackage{hyphens}{url}
%
\documentclass[
]{article}
\usepackage{amsmath,amssymb}
\usepackage{lmodern}
\usepackage{iftex}
\ifPDFTeX
  \usepackage[T1]{fontenc}
  \usepackage[utf8]{inputenc}
  \usepackage{textcomp} % provide euro and other symbols
\else % if luatex or xetex
  \usepackage{unicode-math}
  \defaultfontfeatures{Scale=MatchLowercase}
  \defaultfontfeatures[\rmfamily]{Ligatures=TeX,Scale=1}
\fi
% Use upquote if available, for straight quotes in verbatim environments
\IfFileExists{upquote.sty}{\usepackage{upquote}}{}
\IfFileExists{microtype.sty}{% use microtype if available
  \usepackage[]{microtype}
  \UseMicrotypeSet[protrusion]{basicmath} % disable protrusion for tt fonts
}{}
\makeatletter
\@ifundefined{KOMAClassName}{% if non-KOMA class
  \IfFileExists{parskip.sty}{%
    \usepackage{parskip}
  }{% else
    \setlength{\parindent}{0pt}
    \setlength{\parskip}{6pt plus 2pt minus 1pt}}
}{% if KOMA class
  \KOMAoptions{parskip=half}}
\makeatother
\usepackage{xcolor}
\usepackage[margin=1in]{geometry}
\usepackage{color}
\usepackage{fancyvrb}
\newcommand{\VerbBar}{|}
\newcommand{\VERB}{\Verb[commandchars=\\\{\}]}
\DefineVerbatimEnvironment{Highlighting}{Verbatim}{commandchars=\\\{\}}
% Add ',fontsize=\small' for more characters per line
\usepackage{framed}
\definecolor{shadecolor}{RGB}{248,248,248}
\newenvironment{Shaded}{\begin{snugshade}}{\end{snugshade}}
\newcommand{\AlertTok}[1]{\textcolor[rgb]{0.94,0.16,0.16}{#1}}
\newcommand{\AnnotationTok}[1]{\textcolor[rgb]{0.56,0.35,0.01}{\textbf{\textit{#1}}}}
\newcommand{\AttributeTok}[1]{\textcolor[rgb]{0.77,0.63,0.00}{#1}}
\newcommand{\BaseNTok}[1]{\textcolor[rgb]{0.00,0.00,0.81}{#1}}
\newcommand{\BuiltInTok}[1]{#1}
\newcommand{\CharTok}[1]{\textcolor[rgb]{0.31,0.60,0.02}{#1}}
\newcommand{\CommentTok}[1]{\textcolor[rgb]{0.56,0.35,0.01}{\textit{#1}}}
\newcommand{\CommentVarTok}[1]{\textcolor[rgb]{0.56,0.35,0.01}{\textbf{\textit{#1}}}}
\newcommand{\ConstantTok}[1]{\textcolor[rgb]{0.00,0.00,0.00}{#1}}
\newcommand{\ControlFlowTok}[1]{\textcolor[rgb]{0.13,0.29,0.53}{\textbf{#1}}}
\newcommand{\DataTypeTok}[1]{\textcolor[rgb]{0.13,0.29,0.53}{#1}}
\newcommand{\DecValTok}[1]{\textcolor[rgb]{0.00,0.00,0.81}{#1}}
\newcommand{\DocumentationTok}[1]{\textcolor[rgb]{0.56,0.35,0.01}{\textbf{\textit{#1}}}}
\newcommand{\ErrorTok}[1]{\textcolor[rgb]{0.64,0.00,0.00}{\textbf{#1}}}
\newcommand{\ExtensionTok}[1]{#1}
\newcommand{\FloatTok}[1]{\textcolor[rgb]{0.00,0.00,0.81}{#1}}
\newcommand{\FunctionTok}[1]{\textcolor[rgb]{0.00,0.00,0.00}{#1}}
\newcommand{\ImportTok}[1]{#1}
\newcommand{\InformationTok}[1]{\textcolor[rgb]{0.56,0.35,0.01}{\textbf{\textit{#1}}}}
\newcommand{\KeywordTok}[1]{\textcolor[rgb]{0.13,0.29,0.53}{\textbf{#1}}}
\newcommand{\NormalTok}[1]{#1}
\newcommand{\OperatorTok}[1]{\textcolor[rgb]{0.81,0.36,0.00}{\textbf{#1}}}
\newcommand{\OtherTok}[1]{\textcolor[rgb]{0.56,0.35,0.01}{#1}}
\newcommand{\PreprocessorTok}[1]{\textcolor[rgb]{0.56,0.35,0.01}{\textit{#1}}}
\newcommand{\RegionMarkerTok}[1]{#1}
\newcommand{\SpecialCharTok}[1]{\textcolor[rgb]{0.00,0.00,0.00}{#1}}
\newcommand{\SpecialStringTok}[1]{\textcolor[rgb]{0.31,0.60,0.02}{#1}}
\newcommand{\StringTok}[1]{\textcolor[rgb]{0.31,0.60,0.02}{#1}}
\newcommand{\VariableTok}[1]{\textcolor[rgb]{0.00,0.00,0.00}{#1}}
\newcommand{\VerbatimStringTok}[1]{\textcolor[rgb]{0.31,0.60,0.02}{#1}}
\newcommand{\WarningTok}[1]{\textcolor[rgb]{0.56,0.35,0.01}{\textbf{\textit{#1}}}}
\usepackage{graphicx}
\makeatletter
\def\maxwidth{\ifdim\Gin@nat@width>\linewidth\linewidth\else\Gin@nat@width\fi}
\def\maxheight{\ifdim\Gin@nat@height>\textheight\textheight\else\Gin@nat@height\fi}
\makeatother
% Scale images if necessary, so that they will not overflow the page
% margins by default, and it is still possible to overwrite the defaults
% using explicit options in \includegraphics[width, height, ...]{}
\setkeys{Gin}{width=\maxwidth,height=\maxheight,keepaspectratio}
% Set default figure placement to htbp
\makeatletter
\def\fps@figure{htbp}
\makeatother
\setlength{\emergencystretch}{3em} % prevent overfull lines
\providecommand{\tightlist}{%
  \setlength{\itemsep}{0pt}\setlength{\parskip}{0pt}}
\setcounter{secnumdepth}{-\maxdimen} % remove section numbering
\ifLuaTeX
  \usepackage{selnolig}  % disable illegal ligatures
\fi
\IfFileExists{bookmark.sty}{\usepackage{bookmark}}{\usepackage{hyperref}}
\IfFileExists{xurl.sty}{\usepackage{xurl}}{} % add URL line breaks if available
\urlstyle{same} % disable monospaced font for URLs
\hypersetup{
  pdftitle={Actividad 2: Modelo SIR},
  pdfauthor={E. Crescio, J. Montalvo, E. Uresti},
  hidelinks,
  pdfcreator={LaTeX via pandoc}}

\title{Actividad 2: Modelo SIR}
\author{E. Crescio, J. Montalvo, E. Uresti}
\date{Noviembre de 2020}

\begin{document}
\maketitle

\hypertarget{el-modelo-sir}{%
\subsection{El modelo SIR}\label{el-modelo-sir}}

Consideremos un modelo para describir la dinámica de un grupo de
individuos de una población con exposición a una enfermedad que puede
contagiarse entre los miembros de la población. Esto puede modelarse
como un sistema dinámico denominado \(SIR\) para una población de \(N\)
individuos en la que se considera la interacción entre un conjunto de
\(S\) individuos \emph{suceptibles} de contraer la enfermedad, un
conjunto \(I\) de individuos \emph{infectados} y uno conjunto \(R\) de
individuos \emph{recuperados} de la enfermedad.

Este modelo tiene los siguientes supuestos:

\begin{itemize}
\item
  la probabilidades de infectarse son iguales para todos los individuos
  de la población;
\item
  la población es homogénea, es decir que los riesgos de infectarse son
  iguales para toos los suceptibles y que los tiempos para recuperarse
  son iguales para todos los infectados; y
\item
  el tamaño \(N\) de la población es constante.
\end{itemize}

El modelo maneja los diferentes conjuntos \(S\), \(I\) y \(R\) como si
fueran compartimentos bien separados y considera que los individuos
pueden pasr de uno a otro en el caso de que se enfermen (cambio
\(S\rightarrow I\)) o que una vez enfermos se recuperen (cambio
\(I\rightarrow\)). Ademas, se asume que un individuo no puede pasar del
conjunto de suceptibles directamente al conjunto de recuperados.

Con estos supuestos y consideraciones, las ecuaciones diferenciales del
modelo SIR son: \[
\begin{aligned}
\frac{dS}{dt}&= -\beta \frac{I}{N} S\\
\frac{dI}{dt}&= \beta\frac{I}{N}S-\gamma I\\\
\frac{dR}{dt}&= \gamma I
\end{aligned}
\] donde:

\begin{itemize}
\item
  N=S+R+I
\item
  la cantidad \(\beta\frac{I}{N}\) representa la razón con que las
  personas salen del compartimento S (se infectan);
\item
  en la primera ecuación \(dS\) representa el cambio debido a las
  personas que salen del compartimento \(S\) (el signo negativo se debe
  a que las personas salen)
\item
  en la segunda ecuación \(dI\) representa el cambio debido a las
  personas que salen del compartimento \(I\) (una parte se debe a las
  personas que del compartimento \(S\) pasan al compartimento \(I\), y
  otra parte se debe a las personas que salen del compartimento \(I\)
  porque se recuperan);
\item
  la cantidad \(\gamma\) representa la razón con que las personas se
  recuperan.
\end{itemize}

\begin{Shaded}
\begin{Highlighting}[]
\CommentTok{\# PACKAGES:}
\FunctionTok{library}\NormalTok{(deSolve)}
\FunctionTok{library}\NormalTok{(reshape2)}
\CommentTok{\#library(ggplot2)}

\NormalTok{hh}\OtherTok{=}\ControlFlowTok{function}\NormalTok{(beta,gamma,b,mu,t,v)\{}
\NormalTok{initial\_state\_values }\OtherTok{\textless{}{-}} \FunctionTok{c}\NormalTok{(}\AttributeTok{S =} \DecValTok{999999}\NormalTok{,  }\CommentTok{\# Número de susceptibles inicial}
                                       \CommentTok{\# }
                          \AttributeTok{I =} \DecValTok{1}\NormalTok{,       }\CommentTok{\# Se inicia con una persona infectada}
                          \AttributeTok{R =} \DecValTok{0}\NormalTok{)       }\CommentTok{\# }


\CommentTok{\#razones en unidades de días\^{}{-}1}
\NormalTok{parameters }\OtherTok{\textless{}{-}} \FunctionTok{c}\NormalTok{(beta,gamma,b,mu,v)   }\CommentTok{\# razón de recuperación}

\CommentTok{\#valores de tiempo para resolver la ecuación, de 0 a 60 días}
\NormalTok{times }\OtherTok{\textless{}{-}} \FunctionTok{seq}\NormalTok{(}\AttributeTok{from =} \DecValTok{0}\NormalTok{, }\AttributeTok{to =}\NormalTok{ t, }\AttributeTok{by =} \FloatTok{0.001}\NormalTok{)   }

\NormalTok{sir\_model }\OtherTok{\textless{}{-}} \ControlFlowTok{function}\NormalTok{(time, state, parameters) \{  }
    \FunctionTok{with}\NormalTok{(}\FunctionTok{as.list}\NormalTok{(}\FunctionTok{c}\NormalTok{(state, parameters)), \{}\CommentTok{\# R obtendrá los nombres de variables a}
                                         \CommentTok{\# partir de inputs de estados y parametros}
\NormalTok{        N }\OtherTok{\textless{}{-}}\NormalTok{ S}\SpecialCharTok{+}\NormalTok{I}\SpecialCharTok{+}\NormalTok{R }
\NormalTok{        lambda }\OtherTok{\textless{}{-}}\NormalTok{ beta }\SpecialCharTok{*}\NormalTok{ I}\SpecialCharTok{/}\NormalTok{N}
\NormalTok{        dS }\OtherTok{\textless{}{-}} \SpecialCharTok{{-}}\NormalTok{lambda }\SpecialCharTok{*}\NormalTok{ S}\SpecialCharTok{+}\NormalTok{b}\SpecialCharTok{*}\NormalTok{N}\SpecialCharTok{{-}}\NormalTok{mu}\SpecialCharTok{*}\NormalTok{S}\SpecialCharTok{{-}}\NormalTok{v}\SpecialCharTok{*}\NormalTok{S           }
\NormalTok{        dI }\OtherTok{\textless{}{-}}\NormalTok{ lambda }\SpecialCharTok{*}\NormalTok{ S }\SpecialCharTok{{-}}\NormalTok{ gamma }\SpecialCharTok{*}\NormalTok{ I}\SpecialCharTok{{-}}\NormalTok{mu}\SpecialCharTok{*}\NormalTok{I }
\NormalTok{        dR }\OtherTok{\textless{}{-}}\NormalTok{ gamma }\SpecialCharTok{*}\NormalTok{ I}\SpecialCharTok{{-}}\NormalTok{mu}\SpecialCharTok{*}\NormalTok{R}\SpecialCharTok{+}\NormalTok{v}\SpecialCharTok{*}\NormalTok{S         }
        \FunctionTok{return}\NormalTok{(}\FunctionTok{list}\NormalTok{(}\FunctionTok{c}\NormalTok{(dS, dI, dR))) }
\NormalTok{    \})}
\NormalTok{\}}

\CommentTok{\# poner la solución del sistema de ecuaciones en forma de un dataframe}
\NormalTok{output }\OtherTok{\textless{}{-}} \FunctionTok{as.data.frame}\NormalTok{(}\FunctionTok{ode}\NormalTok{(}\AttributeTok{y =}\NormalTok{ initial\_state\_values, }
                            \AttributeTok{times =}\NormalTok{ times, }
                            \AttributeTok{func =}\NormalTok{ sir\_model,}
                            \AttributeTok{parms =}\NormalTok{ parameters))}
\FunctionTok{return}\NormalTok{(output)}
\NormalTok{\}}
\end{Highlighting}
\end{Shaded}

\hypertarget{gruxe1ficos-de-la-evoluciuxf3n-del-sistema}{%
\subsection{Gráficos de la evolución del
sistema}\label{gruxe1ficos-de-la-evoluciuxf3n-del-sistema}}

\#```\{r \} output\_long \textless- melt(as.data.frame(output), id =
``time'')

ggplot(data = output\_long,\\
aes(x = time, y = value, colour = variable, group = variable)) +\\
geom\_line() +\\
xlab(``Tiempo (días)'')+\\
ylab(``Número de individuos'') +\\
labs(colour = ``Subconjunto'') + theme(legend.position = ``bottom'')
\#```

Con el modelo SIR se define la constante \[R_0=\frac{\beta}{\gamma}\]
que representa el número de personas que cada contagiado infecta. Para
que la enfermedad analizada logre dispararse en forma de una epidemia
debe cumplirse que \(R_0 > 1\).

También se define \[R_{eff}=R_0\frac{S}{N}\] que corresponde al número
promedio de personas que cada contagiado infecta. Este segundo valor
\(R_{eff}\) toma en cuenta de que durante la evolución de la pandemia,
al aumentar del número de personas inmunes en la población cada persona
contagiada infectará a un número de personas cada vez menor.

\hypertarget{pregunta-1}{%
\subsection{Pregunta 1}\label{pregunta-1}}

Haga cambios en el modelo para tomar en cuenta el hecho de que la
población no es constante:

\begin{itemize}
\item
  agregar un término de incremento en \(dS\) para tomar en cuenta los
  individuos nacidos \(+bN\)
\item
  agregar un término de decremento en \(dS\) para tomar en cuenta las
  personas susceptibles que mueren -\(\mu S\)
\item
  agregar un término de decremento en \(dI\) para tomar en cuenta las
  personas infectadas que mueren -\(\mu I\)
\item
  agregar un término de decremento en \(dR\) para tomar en cuenta las
  personas recuperadas que fallecen \(-\mu R\)
\end{itemize}

Usar ahora los paámetros \[
\begin{aligned}
\beta  &=  0.4 days^{-1} &= (0.4 \times 365) years^{-1}\\
\gamma &=  0.2 days^{-1} &= (0.2 \times 365) years^{-1}\\
\mu    &=  \frac{1}{70}years^{-1}\\
b     &=  \frac{1}{70}years^{-1}\\
\end{aligned}
\] y considerar una duración de 1 año.

\[
\begin{aligned}
\frac{dS}{dt}&= -\beta \frac{I}{N} S+bN-\mu S-vS\\
\frac{dI}{dt}&= \beta\frac{I}{N}S-\gamma I-\mu I\\\
\frac{dR}{dt}&= \gamma I-\mu R+vS
\end{aligned}
\]

\begin{Shaded}
\begin{Highlighting}[]
\NormalTok{output}\OtherTok{=}\FunctionTok{hh}\NormalTok{(}\AttributeTok{beta =} \FloatTok{0.4}\SpecialCharTok{*}\DecValTok{365}\NormalTok{,}\AttributeTok{gamma =} \FloatTok{0.2}\SpecialCharTok{*}\DecValTok{365}\NormalTok{,}\AttributeTok{b=}\DecValTok{1}\SpecialCharTok{/}\DecValTok{70}\NormalTok{,}\AttributeTok{mu=}\DecValTok{1}\SpecialCharTok{/}\DecValTok{70}\NormalTok{,}\AttributeTok{t=}\DecValTok{1}\NormalTok{,}\AttributeTok{v=}\DecValTok{0}\NormalTok{)}
\FunctionTok{plot}\NormalTok{(output[,}\DecValTok{1}\NormalTok{],output[,}\DecValTok{2}\NormalTok{],}\AttributeTok{type=}\StringTok{"l"}\NormalTok{,}\AttributeTok{main=}\StringTok{"S"}\NormalTok{,}\AttributeTok{col=}\StringTok{"red"}\NormalTok{,}\AttributeTok{lwd=}\FloatTok{1.5}\NormalTok{)}
\FunctionTok{par}\NormalTok{(}\AttributeTok{new=}\ConstantTok{TRUE}\NormalTok{)}
\FunctionTok{plot}\NormalTok{(output[,}\DecValTok{1}\NormalTok{],output[,}\DecValTok{3}\NormalTok{],}\AttributeTok{type=}\StringTok{"l"}\NormalTok{,}\AttributeTok{main=}\StringTok{"I"}\NormalTok{,}\AttributeTok{col=}\StringTok{"green"}\NormalTok{,}\AttributeTok{lwd=}\FloatTok{1.5}\NormalTok{)}
\FunctionTok{par}\NormalTok{(}\AttributeTok{new=}\ConstantTok{TRUE}\NormalTok{)}
\FunctionTok{plot}\NormalTok{(output[,}\DecValTok{1}\NormalTok{],output[,}\DecValTok{4}\NormalTok{],}\AttributeTok{type=}\StringTok{"l"}\NormalTok{,}\AttributeTok{main=}\StringTok{"R"}\NormalTok{,}\AttributeTok{col=}\StringTok{"blue"}\NormalTok{,}\AttributeTok{lwd=}\FloatTok{1.5}\NormalTok{)}
\end{Highlighting}
\end{Shaded}

\includegraphics{MA1002B_Actividad2_files/figure-latex/unnamed-chunk-2-1.pdf}

\hypertarget{pregunta-2}{%
\subsection{Pregunta 2}\label{pregunta-2}}

Considerando el modelo SIR básico, haga cambios para tomar en cuenta un
programa de vacunación. Suponga que una fracción \(v\) de susceptibles
se vacuna de manera que queda inmune (y entra ahora directamente en el
conjunto de los recuperados). Calcule la dinámica de la epidemia en este
caso usando los parámetros \(\beta=0.4\), \(\gamma=0.1\) y considere un
periodo de 2 años.

Su modelo debe ser capaz de mostrar que si la fracción \(v\) es
suficiente, no es necesario vacunar a todos los suceptibles para evitar
la epidemia. A este efecto se le conoce como \emph{inmunidad de rebaño}
y se refiere a que si un sector grande de la población es inmune,
entonces los contagios se mantienen a un nivel en el que la enfermedad
es eliminada.

¿Cómo se puede calcular la fracción mínima \(v\) de personas que se
deben vacunar para poder evitar una epidemia? La inmunidad de rebaño
ocurre cuando \(R_{eff}< 1\).

\[
\begin{aligned}
\frac{dS}{dt}&= -\beta \frac{I}{N} S-vS\\
\frac{dI}{dt}&= \beta\frac{I}{N}S-\gamma I\\\
\frac{dR}{dt}&= \gamma I-\mu R+vS
\end{aligned}
\]

\$\$\$\$

\[
\\
R_{eff}=R_0\frac{S}{N}
\\
R_0=\frac{\beta}{\gamma}=\frac{0.4}{0.1}=4
\\
1>R_0\frac{S}{N}
\\
\frac{N}{R_0}>S
\\
\frac{100,000}{4}=25,000>S
\\
v=\frac{25,000}{99,999}\approx\frac{1}{4} \therefore v<\frac{1}{4}
\]

\begin{Shaded}
\begin{Highlighting}[]
\NormalTok{g}\OtherTok{=}\FloatTok{0.1}
\NormalTok{b}\OtherTok{=}\FloatTok{0.4}
\NormalTok{R\_0 }\OtherTok{=}\NormalTok{b}\SpecialCharTok{/}\NormalTok{g}

\NormalTok{output}\OtherTok{=}\FunctionTok{hh}\NormalTok{(}\AttributeTok{beta =}\NormalTok{ b}\SpecialCharTok{*}\DecValTok{365}\NormalTok{,}\AttributeTok{gamma =}\NormalTok{ g}\SpecialCharTok{*}\DecValTok{365}\NormalTok{,}\AttributeTok{b=}\DecValTok{1}\SpecialCharTok{/}\DecValTok{70}\NormalTok{,}\AttributeTok{mu=}\DecValTok{1}\SpecialCharTok{/}\DecValTok{70}\NormalTok{,}\AttributeTok{t=}\DecValTok{2}\NormalTok{,}\AttributeTok{v=}\DecValTok{1}\SpecialCharTok{/}\DecValTok{4}\NormalTok{)}
\CommentTok{\#N=output[,2]+output[,3]+output[,4]}
\CommentTok{\#plot(output[,1],R\_0*output[,2]/N,main="Reff",type="l",col="black",lwd="4")}
\FunctionTok{plot}\NormalTok{(output[,}\DecValTok{1}\NormalTok{],output[,}\DecValTok{2}\NormalTok{],}\AttributeTok{type=}\StringTok{"l"}\NormalTok{,}\AttributeTok{main=}\StringTok{"S"}\NormalTok{,}\AttributeTok{col=}\StringTok{"red"}\NormalTok{,}\AttributeTok{lwd=}\FloatTok{1.5}\NormalTok{)}
\FunctionTok{par}\NormalTok{(}\AttributeTok{new=}\ConstantTok{TRUE}\NormalTok{)}
\FunctionTok{plot}\NormalTok{(output[,}\DecValTok{1}\NormalTok{],output[,}\DecValTok{3}\NormalTok{],}\AttributeTok{type=}\StringTok{"l"}\NormalTok{,}\AttributeTok{main=}\StringTok{"I"}\NormalTok{,}\AttributeTok{col=}\StringTok{"green"}\NormalTok{,}\AttributeTok{lwd=}\FloatTok{1.5}\NormalTok{)}
\FunctionTok{par}\NormalTok{(}\AttributeTok{new=}\ConstantTok{TRUE}\NormalTok{)}
\FunctionTok{plot}\NormalTok{(output[,}\DecValTok{1}\NormalTok{],output[,}\DecValTok{4}\NormalTok{],}\AttributeTok{type=}\StringTok{"l"}\NormalTok{,}\AttributeTok{main=}\StringTok{"R"}\NormalTok{,}\AttributeTok{col=}\StringTok{"blue"}\NormalTok{,}\AttributeTok{lwd=}\FloatTok{1.5}\NormalTok{)}
\end{Highlighting}
\end{Shaded}

\includegraphics{MA1002B_Actividad2_files/figure-latex/unnamed-chunk-3-1.pdf}

\hypertarget{pregunta-3}{%
\subsection{Pregunta 3}\label{pregunta-3}}

Haga cambios en el modelo para tomar en cuenta de que la población no es
constante:

\begin{itemize}
\item
  agregar un término de incremento en \(dS\) para tomar en cuenta los
  nacidos \(+bN\)
\item
  agregar un término de decremento en \(dS\) para tomar en cuenta las
  personas susceptibles que mueren -\(\mu S\)
\item
  agregar un término de decremento en \(dI\) para tomar en cuenta las
  personas infectadas que mueren -\(\mu I\)
\item
  agregar un término de decremento en \(dR\) para tomar en cuenta las
  personas recuperadas que fallecen \(-\mu R\)
\end{itemize}

Use los parámetros \[
\begin{aligned}
\beta  &=  0.4 days^{-1} &= (0.4 \times 365) years^{-1}\\
\gamma &=  0.2 days^{-1} &= (0.2 \times 365) years^{-1}\\
\mu    &=  \frac{1}{70}years^{-1}\\
b     &=  \frac{1}{70}years^{-1}\\
\end{aligned}
\] y considere una duración de 400 años en sus cálculos.

\begin{Shaded}
\begin{Highlighting}[]
\NormalTok{output}\OtherTok{=}\FunctionTok{hh}\NormalTok{(}\AttributeTok{beta =} \FloatTok{0.4}\SpecialCharTok{*}\DecValTok{365}\NormalTok{,}\AttributeTok{gamma =} \FloatTok{0.2}\SpecialCharTok{*}\DecValTok{365}\NormalTok{,}\AttributeTok{b=}\DecValTok{1}\SpecialCharTok{/}\DecValTok{70}\NormalTok{,}\AttributeTok{mu=}\DecValTok{1}\SpecialCharTok{/}\DecValTok{70}\NormalTok{,}\AttributeTok{t=}\DecValTok{400}\NormalTok{,}\AttributeTok{v=}\DecValTok{0}\NormalTok{)}
\FunctionTok{plot}\NormalTok{(output[,}\DecValTok{1}\NormalTok{],output[,}\DecValTok{2}\NormalTok{],}\AttributeTok{type=}\StringTok{"l"}\NormalTok{,}\AttributeTok{main=}\StringTok{"S"}\NormalTok{,}\AttributeTok{col=}\StringTok{"red"}\NormalTok{,}\AttributeTok{lwd=}\FloatTok{1.5}\NormalTok{)}
\FunctionTok{par}\NormalTok{(}\AttributeTok{new=}\ConstantTok{TRUE}\NormalTok{)}
\FunctionTok{plot}\NormalTok{(output[,}\DecValTok{1}\NormalTok{],output[,}\DecValTok{3}\NormalTok{],}\AttributeTok{type=}\StringTok{"l"}\NormalTok{,}\AttributeTok{main=}\StringTok{"I"}\NormalTok{,}\AttributeTok{col=}\StringTok{"green"}\NormalTok{,}\AttributeTok{lwd=}\FloatTok{1.5}\NormalTok{)}
\FunctionTok{par}\NormalTok{(}\AttributeTok{new=}\ConstantTok{TRUE}\NormalTok{)}
\FunctionTok{plot}\NormalTok{(output[,}\DecValTok{1}\NormalTok{],output[,}\DecValTok{4}\NormalTok{],}\AttributeTok{type=}\StringTok{"l"}\NormalTok{,}\AttributeTok{main=}\StringTok{"R"}\NormalTok{,}\AttributeTok{col=}\StringTok{"blue"}\NormalTok{,}\AttributeTok{lwd=}\FloatTok{1.5}\NormalTok{)}
\end{Highlighting}
\end{Shaded}

\includegraphics{MA1002B_Actividad2_files/figure-latex/unnamed-chunk-4-1.pdf}

\hypertarget{pregunta-4}{%
\subsection{Pregunta 4}\label{pregunta-4}}

Considerando el modelo SIR básico, haga cambios para tomar en cuenta un
programa de vacunación. Suponga que una fracción \(v\) de susceptibles
se vacuna de manera que queda inmune (y entra ahora directamente en el
conjunto de los recuperados), mientras que la fracción \((1-v)\) sigue
siendo susceptible.

Calcule la dinámica de la epidemia en este caso, estudiando cómo cambia
la dinámica variando la fracción \(v\). Utilice \(\beta=0.6\),
\(\gamma=0.1\) y considere un periodo de 2 años.

Su modelo debe ser capaz de mostrar que si la fracción \(v\) es
suficiente, no es necesario vacunar a todos los suceptibles para evitar
la epidemia. A este efecto se le conoce como \emph{inmunidad de rebaño}
y se refiere a que si un sector grande de la población es inmune,
entonces los contagios se mantienen a un nivel en el que la enfermedad
es eliminada.

¿Cómo se puede calcular la fracción mínima \(v\) de personas que se
deben vacunar para poder evitar una epidemia? La inmunidad de rebaño
ocurre cuando \(R_{eff}< 1\).

\[
\begin{aligned}
\frac{dS}{dt}&= -\beta \frac{I}{N} S+bN-\mu S-(1-v)S\\
\frac{dI}{dt}&= \beta\frac{I}{N}S-\gamma I-\mu I\\\
\frac{dR}{dt}&= \gamma I-\mu R+vS
\end{aligned}
\]

\begin{Shaded}
\begin{Highlighting}[]
\CommentTok{\# PACKAGES:}
\FunctionTok{library}\NormalTok{(deSolve)}
\FunctionTok{library}\NormalTok{(reshape2)}
\CommentTok{\#library(ggplot2)}

\NormalTok{hhh}\OtherTok{=}\ControlFlowTok{function}\NormalTok{(beta,gamma,b,mu,t,v)\{}
\NormalTok{initial\_state\_values }\OtherTok{\textless{}{-}} \FunctionTok{c}\NormalTok{(}\AttributeTok{S =} \DecValTok{999999}\NormalTok{,  }\CommentTok{\# Número de susceptibles inicial}
                                       \CommentTok{\# }
                          \AttributeTok{I =} \DecValTok{1}\NormalTok{,       }\CommentTok{\# Se inicia con una persona infectada}
                          \AttributeTok{R =} \DecValTok{0}\NormalTok{)       }\CommentTok{\# }


\CommentTok{\#razones en unidades de días\^{}{-}1}
\NormalTok{parameters }\OtherTok{\textless{}{-}} \FunctionTok{c}\NormalTok{(beta,gamma,b,mu,v)   }\CommentTok{\# razón de recuperación}

\CommentTok{\#valores de tiempo para resolver la ecuación, de 0 a 60 días}
\NormalTok{times }\OtherTok{\textless{}{-}} \FunctionTok{seq}\NormalTok{(}\AttributeTok{from =} \DecValTok{0}\NormalTok{, }\AttributeTok{to =}\NormalTok{ t, }\AttributeTok{by =} \FloatTok{0.001}\NormalTok{)   }

\NormalTok{sir\_model }\OtherTok{\textless{}{-}} \ControlFlowTok{function}\NormalTok{(time, state, parameters) \{  }
    \FunctionTok{with}\NormalTok{(}\FunctionTok{as.list}\NormalTok{(}\FunctionTok{c}\NormalTok{(state, parameters)), \{}\CommentTok{\# R obtendrá los nombres de variables a}
                                         \CommentTok{\# partir de inputs de estados y parametros}
\NormalTok{        N }\OtherTok{\textless{}{-}}\NormalTok{ S}\SpecialCharTok{+}\NormalTok{I}\SpecialCharTok{+}\NormalTok{R }
\NormalTok{        lambda }\OtherTok{\textless{}{-}}\NormalTok{ beta }\SpecialCharTok{*}\NormalTok{ I}\SpecialCharTok{/}\NormalTok{N}
\NormalTok{        dS }\OtherTok{\textless{}{-}} \SpecialCharTok{{-}}\NormalTok{lambda }\SpecialCharTok{*}\NormalTok{ S}\SpecialCharTok{+}\NormalTok{b}\SpecialCharTok{*}\NormalTok{N}\SpecialCharTok{{-}}\NormalTok{mu}\SpecialCharTok{*}\NormalTok{S}\SpecialCharTok{{-}}\NormalTok{(}\DecValTok{1}\SpecialCharTok{{-}}\NormalTok{v)}\SpecialCharTok{*}\NormalTok{S           }
\NormalTok{        dI }\OtherTok{\textless{}{-}}\NormalTok{ lambda }\SpecialCharTok{*}\NormalTok{ S }\SpecialCharTok{{-}}\NormalTok{ gamma }\SpecialCharTok{*}\NormalTok{ I}\SpecialCharTok{{-}}\NormalTok{mu}\SpecialCharTok{*}\NormalTok{I }
\NormalTok{        dR }\OtherTok{\textless{}{-}}\NormalTok{ gamma }\SpecialCharTok{*}\NormalTok{ I}\SpecialCharTok{{-}}\NormalTok{mu}\SpecialCharTok{*}\NormalTok{R}\SpecialCharTok{+}\NormalTok{v}\SpecialCharTok{*}\NormalTok{S         }
        \FunctionTok{return}\NormalTok{(}\FunctionTok{list}\NormalTok{(}\FunctionTok{c}\NormalTok{(dS, dI, dR))) }
\NormalTok{    \})}
\NormalTok{\}}

\CommentTok{\# poner la solución del sistema de ecuaciones en forma de un dataframe}
\NormalTok{output }\OtherTok{\textless{}{-}} \FunctionTok{as.data.frame}\NormalTok{(}\FunctionTok{ode}\NormalTok{(}\AttributeTok{y =}\NormalTok{ initial\_state\_values, }
                            \AttributeTok{times =}\NormalTok{ times, }
                            \AttributeTok{func =}\NormalTok{ sir\_model,}
                            \AttributeTok{parms =}\NormalTok{ parameters))}
\FunctionTok{return}\NormalTok{(output)}
\NormalTok{\}}
\end{Highlighting}
\end{Shaded}

\begin{Shaded}
\begin{Highlighting}[]
\NormalTok{output}\OtherTok{=}\FunctionTok{hhh}\NormalTok{(}\AttributeTok{beta =} \FloatTok{0.6}\SpecialCharTok{*}\DecValTok{365}\NormalTok{,}\AttributeTok{gamma =} \FloatTok{0.1}\SpecialCharTok{*}\DecValTok{365}\NormalTok{,}\AttributeTok{b=}\DecValTok{1}\SpecialCharTok{/}\DecValTok{70}\NormalTok{,}\AttributeTok{mu=}\DecValTok{1}\SpecialCharTok{/}\DecValTok{70}\NormalTok{,}\AttributeTok{t=}\DecValTok{2}\NormalTok{,}\AttributeTok{v=}\DecValTok{1}\SpecialCharTok{/}\DecValTok{4}\NormalTok{)}
\FunctionTok{plot}\NormalTok{(output[,}\DecValTok{1}\NormalTok{],output[,}\DecValTok{2}\NormalTok{],}\AttributeTok{type=}\StringTok{"l"}\NormalTok{,}\AttributeTok{main=}\StringTok{"S"}\NormalTok{,}\AttributeTok{col=}\StringTok{"red"}\NormalTok{,}\AttributeTok{lwd=}\FloatTok{1.5}\NormalTok{)}
\FunctionTok{par}\NormalTok{(}\AttributeTok{new=}\ConstantTok{TRUE}\NormalTok{)}
\FunctionTok{plot}\NormalTok{(output[,}\DecValTok{1}\NormalTok{],output[,}\DecValTok{3}\NormalTok{],}\AttributeTok{type=}\StringTok{"l"}\NormalTok{,}\AttributeTok{main=}\StringTok{"I"}\NormalTok{,}\AttributeTok{col=}\StringTok{"green"}\NormalTok{,}\AttributeTok{lwd=}\FloatTok{1.5}\NormalTok{)}
\FunctionTok{par}\NormalTok{(}\AttributeTok{new=}\ConstantTok{TRUE}\NormalTok{)}
\FunctionTok{plot}\NormalTok{(output[,}\DecValTok{1}\NormalTok{],output[,}\DecValTok{4}\NormalTok{],}\AttributeTok{type=}\StringTok{"l"}\NormalTok{,}\AttributeTok{main=}\StringTok{"R"}\NormalTok{,}\AttributeTok{col=}\StringTok{"blue"}\NormalTok{,}\AttributeTok{lwd=}\FloatTok{1.5}\NormalTok{)}
\end{Highlighting}
\end{Shaded}

\includegraphics{MA1002B_Actividad2_files/figure-latex/unnamed-chunk-6-1.pdf}

\end{document}
